% This is the ADASS_template.tex LaTeX file, 26th August 2016.
% It is based on the ASP general author template file, but modified to reflect the specific
% requirements of the ADASS proceedings.
% Copyright 2014, Astronomical Society of the Pacific Conference Series
% Revision:  14 August 2014

% To compile, at the command line positioned at this folder, type:
% latex ADASS_template
% latex ADASS_template
% dvipdfm ADASS_template
% This will create a file called aspauthor.pdf.}

\documentclass[11pt,twoside]{article}

% Do NOT use ANY packages other than asp2014. 
\usepackage{asp2014}
\usepackage{eurosym}
\aspSuppressVolSlug
\resetcounters

% References must all use BibTeX entries in a .bibfile.
% References must be cited in the text using \citet{} or \citep{}.
% Do not use \cite{}.
% See ManuscriptInstructions.pdf for more details
\bibliographystyle{asp2014}

% The ``markboth'' line sets up the running heads for the paper.
% 1 author: "Surname"
% 2 authors: "Surname1 and Surname2"
% 3 authors: "Surname1, Surname2, and Surname3"
% >3 authors: "Surname1 et al."
% Replace ``Short Title'' with the actual paper title, shortened if necessary.
% Use mixed case type for the shortened title
% Ensure shortened title does not cause an overfull hbox LaTeX error
% See ASPmanual2010.pdf 2.1.4  and ManuscriptInstructions.pdf for more details
\markboth{Rees}{Quality software: what does it mean?}

\begin{document}

\title{Quality software: what does it mean?}

% Note the position of the comma between the author name and the 
% affiliation number.
% Author names should be separated by commas.
% The final author should be preceded by "and".
% Affiliations should not be repeated across multiple \affil commands. If several
% authors share an affiliation this should be in a single \affil which can then
% be referenced for several author names.
% See ManuscriptInstructions.pdf and ASPmanual2010.pdf 3.1.4 for more details
\author{Nick Rees,$^1$
\affil{$^1$SKA Organization, Macclesfield, Cheshire, United Kingdom; 	\email{n.rees@skatelescope.org}}
}

% This section is for ADS Processing.  There must be one line per author.
\paperauthor{Nick Rees}{n.rees@skatelesscope.org}{}{SKA Organization}{Computing and Software}{Macclesfield}{Cheshire}{SK11 9DL}{United Kingdom}

\begin{abstract}
The SKA is a major international project which will generate transformational science in the realm of metre and centimetre astronomy. It can be described as a ``software telescope'', with RF signals being digitized directly and generating aggregated data rates of up to ~1.7 Petabits/sec from the antennas, and intermediate data rates in the signal chain of ~5 Terabits/sec. The telescope is being designed by 10 pre-construction consortia with input from around 250 institutes world-wide and will enter the construction phase in 2019. The consortia are estimating that the computing software and hardware construction costs are both in the \euro80-90M range. 

Many of these numbers are unprecedented in astronomy and this has led to much soul searching. In 2016 the SKA Organization, which coordinates the design consortia, has used software quality, in all its forms, as a way of rationalizing key design decisions and defining primary project processes. Whilst most of this could be seen as being based on best practices elsewhere in the software industry, it has brought a number of fresh perspectives that have not been actively explored in many astronomy projects. The author will present some of these perspectives and how they, along with project risk, are seen as the primary drivers of the software processes and design.
\end{abstract}

\section{Introduction and warning}
I have decided break all the rules and write this paper in a chatty first-person style rather than a formal third person style because it reflects a personal journey over 18 months rather than reporting anything that is new and ground-breaking. Some of the assertions may be provocative but I hope that they are interpreted in the way they are intended - to contribute to debate and understanding - rather any offense being taken. In my defence, whilst what I have to say may be well understood in some communities, I don't believe it is understood by everyone in the astronomical software community and it is unlikely that all aspects of the puzzle can be seen together. It also reflects a view of the top-level status of SKA software development at this point in time, and so this, in itself, also has merit.

\section{What is the problem?}
The senior management of large projects often comprise people who have limited software management experience. The systems they understand typically have limited modifiability requirements after the construction period has completed, so the processes they are familiar with don't emphasise flexibility in this area. There is nothing wrong with this per se, but it does provide a barrier that software managers have to overcome. We may compound the problem by failing to communicate in a language that our managers understand, or our ideas may be poorly formeD -- we know a process is far from ideal, but aren't able to convincingly describe a process which is demonstrably better.

I encountered a similar situation when I arrived at the SKA organisation. There was a large emphasis on documentation based earned value milestones and in some areas of software individuals were working on documentation that they knew was irrelevant because of recent design changes, but still had to be completed to meet some earned value milestone. In fact, whilst I never posed it in this way, I am sure if I asked them where what they were doing related to the Agile Manifesto\cite{agile} they would have concluded they were on the wrong side of each of the four manifesto value statements. In fact, this was so entrenched that you could argue we are still in this position, but I now believe there is some light at the end of the tunnel.

\section{How did we get here?}
When I look back at the management of large ground-based astronomy projects I feel that a pivotal time was in the late 1980's and early 1990's when the first 8-10m class optical telescopes were built. At that time I remember instruments transitioned from relatively small systems that could be built by a university to large systems that needed significant project and system management. As a result there were a number of failures and consequently we had to change our development processes to one better suited to these larger systems.

Since then, I believe that there has been a revolution in software development methods that has led to commercial companies, such as Google, Amazon and the like, being able to consistently build huge software systems that are continually evolving whilst still maintaining a high level of quality. In doing this they have overcome the issues that Conway highlighted in his most famous for his law\cite{conway1968} ``organizations which design systems ... are constrained to produce designs which are copies of the communication structures of these organizations''. The most interesting thing I think about this paper is not the law, but his conclusion, two paragraphs later: ``Ways must be found to reward design managers for keeping their organizations lean and flexible. There is need for a philosophy of system design management which is not based on the assumption that adding manpower simply adds to productivity.''

To a certain extent the Astronomical software community has been aware of all of this, but as mentioned above, we often find we are working in projects that often inherit processes and methods that feel more 1980's than modern day. Having encountered this at the SKA, this paper is about how we have tried to overcome these problems and build a set of software development processes that will be better suited to our core goals.

\section{What did I do about it?}
When I arrived at the SKA I spent some time trying to understand what the most important aspect of my role was to be. I had no experience with spending this amount of money on computer software and hardware and I need some way to bring all the pieces of the puzzle together. 

After a few months I decided that my main focus should be on software quality. At the time I didn't really understand where this journey was going to take me, but I felt it was a reasonable place to start. I started by writing down a few fundamental truths and socialised them around the SKA community. These are the``Fundamental SKA Software and Hardware Definition Language Standard'' and cover things like:
\begin{itemize}
\item Software must have a license.
\item If the software comes off the shelf, we have to have it on a register, it has to meet our needs and we have to understand how we will support it during its lifecycle -- which could be the 50 year life of the SKA telescopes.
\item If the software is developed by the SKA
\begin{itemize}
\item The development must follow a process
\item The process should ensure that the code changes are reviewed and tested.
\item The dependencies should be understood.
\item etc.
\end{itemize}
\end{itemize}

After some time I realised that this whilst necessary, is not sufficient. At one point I called this``tactical'' quality -- the software will at least pass some basic tests, to emphasise that it is separate from a ``strategic'' quality, which is that the software is 

\section{What is quality?}
To understand quality, you need to know what it is. If you ask a software user they can easily describe it by the inverse of some of their experiences - words such as inflexible, fragile, overly complicated, difficult to understand, unpredictable and slow spring to mind. More seriously, the International Standards Organisation have written a standard on it where it describes quality in terms of "quality in use" model and a "product quality" model \citep[See][]{iso25010}. 

\articlefigure{QualityInUse.eps}{i10_fig1}{ISO 25010 Quality in Use model.}

\articlefigure{ProductQuality.eps}{i10_fig2}{ISO 25010 Product Quality model.}

Looking at these it becomes clear that these qualities are a class of requirements that are often called ``non-functional''. 


\section{The implications}


\section{What did I learn?}


\section{Summary and conclusions}



\section{References}


\acknowledgements The ASP would like to thank the dedicated researchers who are publishing with the ASP.  It will make things a lot easier if you keep this text on the same line as the \verb"\acknowledgements" command.


\bibliography{I10}  % For BibTex

\end{document}
